\subsection{Vytvořit záznam letu}
Vytvoření leteckého záznam umožňuje uživateli vložit nový záznam letu do aplikace.

Hlavní scénář --
\begin{itemize}
\item Případ užití začíná, když chce uživatel evidovat svůj let.
\item Systém zobrazí formulář umožňující zadat: jméno velícího pilota, datum letu, čas a místo odletu a příletu, letadlo, čas letu, celkový čas letu, počet vzletů a přistání, svou funkci při letu a provozní podmínky.
\item Uživatel vyplní všechny povinné položky.
\item Aplikace uloží informace o letu.
\end{itemize}

Alternativní scénář --
\begin{itemize}
\item Případ užití začíná, když chce uživatel evidovat záznam z výcvikového zařízení pro simulaci letu.
\item Systém zobrazí formulář  umožňující zadat: typ a kvalifikační číslo výcvikového zařízení, datum a čas.
\item Připad užití pokračuje 3. krokem hlavního scénáře.
\end{itemize}

\subsection{Přidat letadlo}
Přidání letadla davá uživateli možnost přidat letadlo, které pak může vkládat do záznámů o letu.

\begin{itemize}
\item Případ užití začíná, když chce uživatel přidat nové letadlo.
\item Systém zobrazí formulář  umožňující zadat: typ, značku model, variantu, registrační číslo letadla a zda je letadlo jednomotorové nebo vícemotorové.
\item Uživatel vyplní všechna pole formuláře.
\item Aplikace uloží letadlo.
\end{itemize}

\subsection{Zobrazit letecké záznamy}
Zobrazení leteckých záznamů zobrazuje jednotlivé záznamy v podobě tabulky, kde u každého záznamu jsou vidět základní informcace. Mezi tyto informace patří: místo odletu a přílet, datum a čas letu.

\subsection{Vyhledat a filtrovat letecké záznamy}
Tato funkcionalita umožňuje uživateli vyhledávání a filtrování leteckých záznamů.

\begin{itemize}
\item Případ užití začíná, pokud chce uživatel vyhledat nebo vyfiltrovat letecké záznamy.
\item Include (Zobrazit letecké záznamy)
\item Aplikace zobrazí formůlář, který umožňuje: zadat hledaný text, nastavit zda se jedná o záznam letu nebo o záznam z výcvikového zařízení, zvolit letadlo, nastavit délku letu, zvolit datum příletu a odletu, nastavit uživatelovu funkci pilota a zvolit provozní podmínky.
\item Uživatel vyplní pole podle, kterých chce vyhledávat/filtrovat.
\item Systém zobrazí pouze záznamy odpovídající zvoleným parametrům.
\end{itemize}

\subsection{Smazat letecký záznam}
Smazání leteckého záznamu umožňuje uživateli smazat letecký záznam, který předtím sám vytvořil.

\begin{itemize}
\item Případ užití začíná, když chce uživatel smazat jeden ze svých leteckých záznamů.
\item Include (Zobrazit letecké záznamy).
\item Uživatel si zvolí záznam, který chce smazat.
\item Aplikace zobrazí potvrzovací dialog.
\item Uživatel potvrdí smazání.
\item Aplikace odsraní položku ze seznamu.
\end{itemize}


\subsection{Upravit letecký záznam}
Upravení leteckého záznamu umožňuje uživateli upravit všechny položky zvoleného letecké záznamu.

\begin{itemize}
\item Případ užití začíná, když chce uživatel upravit letecký záznam.
\item Include (Zobrazit letecké záznamy).
\item Uživatel si zvolí záznam, který chce upravit.
\item Scénář pokračuje krokem 2 UC Vytvořit záznam letu.
\end{itemize}

\subsection{Zobrazit limity}
Tato funkcionalita slouží ke zobrazení limitů a kontrole zda jsou všechny limity v normě.

\subsection{Zobrazit certifikáty}
Tato funkcionalita umožňuje uživateli zobrazit všechny jeho certifikáty, společně s kontrolou platnosti a počtem dní do jejich expirace.


\subsection{Přidat certifikát}
Tato funkce umožňuje uživateli přidat certifikát a to buď dle šablony pro zdravotní certifikáty (LALP, třídy 1 a třídy 2), nebo vlastní.

\begin{itemize}
\item Případ užití začíná, když chce uživatel vytvořit nový certifikát.
\item Include (Zobrazit certifikáty).
\item Aplikace zobrazí formůlář s možností vytvoření vlastního certifikátu nebo dle šablony. Ve formuláři je následně možné zadat: název certifikátu, datum vydání a datum expirace a popis.
\item Uživatel povinně vyplní název a datum expirace.
\item Aplikace uloží certifikát.
\end{itemize}

\subsection{Editovat osobní údaje}
Editace osobních údajů umožňuje uživateli upravit své osobní informace. Mezi tyto informace patří: jméno a příjmení, adresa a věk.

\subsection{Generovat report}
Generování reportu umožňuje uživateli vygenerovat report ve formátu PDF z nímž zvolených záznamů.

\begin{itemize}
\item Případ užití začíná, jestliže se uživatel rozhodne vygenerovat report.
\item Aplikace zobrazí formulář s možným výběrem záznamů, které se v reportu objeví.
\item Uživatel si zvolí záznamy.
\item Aplikace vytvoří report ve formátu PDF se zvolenými záznami letů.
\end{itemize}


\subsection{Odeslat report emailem}
Funkcionalita odeslání reportu emailem umožňuje uživateli, poté co vygeneroval report, odeslat tento report přes email.



